%; whizzy -advi "advi -html Start-Document -edit" -mkfile ls

\documentclass{article}

\csname WhizzyShowMasterfalse\endcsname
  
\begin{document}

\section {This is the {\em master} file}
   
This is a test with multiple files.  
This  is the {\bf master} file \verb"main.tex".   
It has two slaves files \verb"first.tex" and \verb"second.tex". 
This is the fist section that does not input any file.
Move to next section for file inclusion.

\section {A master section, loading the first slave} 

The rest of this section is input from \texttt{first.tex}.
The rest of this section ends with:
\begin{verbatim}
%; whizzy-master main.tex


aaa aaaa

\subsection {This is the first (sub)section of the slave}

This file {\tt first.tex} is mastered by {\tt main.tex}.  The slice is at
most the current file (but may be smaller).  The slice is surrounded by
titles reminding the master, and written in such a way that positioning on
these in the previwer (if you are previeweing with advi) should point to the
master file (at a non-existing position interpretted as the old position).
 

You can edit the slave, move in the slave, or position in the previewer. 

You can also return to the master from emacs or by clicking on the master
text in the previewer. 

\section {Another slice in the first slave}

The slave will only show its own slice (and the surronding master marks)

The slace can input another slace

\subsection {An inlined file inside an include}

We are now in a file loaded by a slave.

And we are are the end of this file.


Now we are back.

   
\end{verbatim}
%; whizzy-master main.tex


aaa aaaa

\subsection {This is the first (sub)section of the slave}

This file {\tt first.tex} is mastered by {\tt main.tex}.  The slice is at
most the current file (but may be smaller).  The slice is surrounded by
titles reminding the master, and written in such a way that positioning on
these in the previwer (if you are previeweing with advi) should point to the
master file (at a non-existing position interpretted as the old position).
 

You can edit the slave, move in the slave, or position in the previewer. 

You can also return to the master from emacs or by clicking on the master
text in the previewer. 

\section {Another slice in the first slave}

The slave will only show its own slice (and the surronding master marks)

The slace can input another slace

\subsection {An inlined file inside an include}

We are now in a file loaded by a slave.

And we are are the end of this file.


Now we are back.

   


 
\section {Another master section, loading the first slave}

The next example is a few sections from an included file (hence the rest of
the slice follows on the next page). Note that, even though slices are by
section the included file, which contained several section is not sliced.
You may click on one of the included sections, which will move the pointer
and pass contol to the included file. Then, only one section of the
included file will be sliced. The sliced is delimited with lines: 
$$
\centerline{\hrulefill\space \lower 0.5ex \hbox{\bf Mastered by main.tex}
\hrulefill}
$$
on which you click to return to the master. 

The rest of this section ends with:
\begin{verbatim}
%; whizzy-master main.tex


\section {Another one, from the second slave}


\subsection {This is the second slave}

This is a test with multiple files.


If I edit the slave, it still works.

The master file is \verb"main.tex"

\section {Another slice in the slave}

The slave will only show its own slice and even 

\section {Another slice in the slave}



\end{verbatim}
%; whizzy-master main.tex


\section {Another one, from the second slave}


\subsection {This is the second slave}

This is a test with multiple files.


If I edit the slave, it still works.

The master file is \verb"main.tex"

\section {Another slice in the slave}

The slave will only show its own slice and even 

\section {Another slice in the slave}




\section {Last section, from master}

You may also input a file from a subdirectory: 
The rest of this section ends with:
\begin{verbatim}
\expandafter\ifx\csname SourceFile\endcsname\relax\else\SourceFile{subfile.ltx}\fi
\expandafter\ifx\csname Setlineno\endcsname\relax\def\Setlineno{\count0}\fi
\Setlineno=0\relax

Another figure

\expandafter\ifx\csname graph\endcsname\relax \csname newbox\endcsname\graph\fi
\expandafter\ifx\csname graphtemp\endcsname\relax \csname newdimen\endcsname\graphtemp\fi
\setbox\graph=\vtop{\vskip 0pt\hbox{%
    \special{pn 8}%
    \special{ar 1250 500 1250 500 0 6.28319}%
    \graphtemp=.5ex\advance\graphtemp by 0.500in
    \rlap{\kern 1.250in\lower\graphtemp\hbox to 0pt{\hss {Group}\hss}}%
    \hbox{\vrule depth1.000in width0pt height 0pt}%
    \kern 2.500in
  }%
}%
\Setlineno=
6
$$
\box \graph
$$

End of figure 

\end{verbatim}
\expandafter\ifx\csname SourceFile\endcsname\relax\else\SourceFile{subfile.ltx}\fi
\expandafter\ifx\csname Setlineno\endcsname\relax\def\Setlineno{\count0}\fi
\Setlineno=0\relax

Another figure

\expandafter\ifx\csname graph\endcsname\relax \csname newbox\endcsname\graph\fi
\expandafter\ifx\csname graphtemp\endcsname\relax \csname newdimen\endcsname\graphtemp\fi
\setbox\graph=\vtop{\vskip 0pt\hbox{%
    \special{pn 8}%
    \special{ar 1250 500 1250 500 0 6.28319}%
    \graphtemp=.5ex\advance\graphtemp by 0.500in
    \rlap{\kern 1.250in\lower\graphtemp\hbox to 0pt{\hss {Group}\hss}}%
    \hbox{\vrule depth1.000in width0pt height 0pt}%
    \kern 2.500in
  }%
}%
\Setlineno=
6
$$
\box \graph
$$

End of figure 


\section{The end}
The document ends here. 


\end{document}

